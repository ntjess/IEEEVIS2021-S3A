\section{Summary of High-Level Impacts}\label{sec:impacts}
%To evaluate these claims against other popular annotation software, an experiment will be carried out to annotate images exhibiting high resolution, numerous foreground regions, and high ROI complexity. Total time required to annotate the images will be compared against the resulting accuracy to show the quantitative effects of the features proposed here.
Key takeaways from \autoref{sec:appFeatures} should strongly emphasize the exploration of annotation capabilities rather than any specific software implementation. Toward this end, S3A's major contributions for annotation and ground truth collection involve meeting both skilled and novice operators at their current level, allowing them to increase the effectiveness of their time during annotation. Beyond this, S3A also supplements existing datasets by providing a unique visualization platform. Its support of dynamic peripheral data, filtering, and more allow operators to quickly hone in on what matters in the current image.

S3A also approaches the task of annotation with a greater emphasis on manual interaction. Throughout the push for increasing automated capabilities, there is still a multitude of insertion points for tweaking parameters and exploring algorithmic solution spaces. As a result, minor adjustments in a given annotation pipeline do not require a reinvention of any individual stage. Rather, subtle changes can be incorporated on an image-by-image or region-by-region basis to leverage the greatest possible output accuracy.