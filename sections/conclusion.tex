\section{Conclusion and Future Work}
The Semi-Supervised Semantic Annotator (S3A) is proposed to address the difficult task of pixel-level annotations of image data. For high-resolution images with numerous complex regions of interest, existing labeling software faces performance bottlenecks attempting to extract ground-truth information. Moreover, there is a lack of capabilities to convert such a labeling workflow into an automated procedure with feedback at every step. Each of these challenges are overcome by various features within S3A specifically designed for such tasks. As a result, S3A provides not only tremendous time savings during ground truth annotation, but also allows an annotation pipeline to be directly converted into a prediction scheme. Furthermore, the rapid feedback accessible at every stage of annotation expedites prototyping of novel solutions to imaging domains in which few examples of prior work exist.

Improvements are ongoing and will include a variety of powerful features. Chief among these will be an algorithm builder to allow greater flexibility with mixing and matching useful region modification stages. Moreover, a minimap, improved robust region selection, more default multi-prediction options, and more export options will both assist S3A's capabilities when handling large images and increase its viability as a professional labeling tool.